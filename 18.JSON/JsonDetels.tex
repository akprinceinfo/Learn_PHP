
JSON (JavaScript Object Notation) হলো একটি লাইটওয়েট ডেটা ইনটারচেঞ্জ ফরম্যাট, যা মানুষ ও মেশিনদের জন্য সহজে বুঝা এবং তৈরি করা যায়। এটি টেক্সট ফরম্যাট এবং কোড এবং ডেটা এর মধ্যে একটি একক স্ট্যান্ডার্ড ইন্টারফেস প্রদান করে। JSON স্ট্রিং হিসেবে রুপান্তরিত করা যায় যেখানে ডেটা গুলি অবজেক্ট বা অ্যারে হিসেবে থাকতে পারে।

বিস্তারিত বর্ণনা:
1. ডেটা টাইপ:
JSON তে প্রিমিটিভ ডেটা টাইপ গুলি হলো -

স্ট্রিং (String): টেক্সট। উদাহরণ: "Hello, World!"
সংখ্যা (Number): কোনো সংখ্যা। উদাহরণ: 42 অথবা 3.14
বুলিয়ান (Boolean): true অথবা false
নাল (Null): null

2. অবজেক্ট (Object):
অবজেক্ট হলো কী-ভ্যালু এর জোড়াসহ একটি কলেকশন। কী হলো একটি স্ট্রিং এবং এটি ভ্যালুকে সাথে জোড়া করে। অবজেক্টের শুরু হয় '{' এবং শেষ হয় '}' এর মধ্যে। কলন সাথে একটি অবস্থানের সিমুলেট হয়, সেটি নিজেই একটি অবজেক্ট হিসেবে ফাঁকা থাকতে পারে। উদাহরণ:

{
  "name": "John",
  "age": 30,
  "city": "New York"
}

3. অ্যারে (Array):
অ্যারে হলো একটি ডেটা কলেকশন যা সিরিজে এবং ইনডেক্সড ডেটা হিসেবে থাকে। অ্যারের শুরু হয় '[' এবং শেষ হয় ']' এর মধ্যে। উদাহরণ:

["apple", "orange", "banana"]

4. নেস্টেড JSON:
JSON এ একটি অবজেক্ট বা অ্যারে অন্য একটি অবজেক্ট বা অ্যারে এবং এমনকি তাদের সমন্বিত ব্যবহার করা যায়, এটি হতে পারে একে অপরের মধ্যে নেস্টেড। উদাহরণ:

{
  "person": {
    "name": "Alice",
    "age": 25,
    "address": {
      "city": "London",
      "country": "UK"
    }
  }
}

5. কমেন্ট:
JSON স্ট্যান্ডার্ড এ কোনো কমেন্ট সাপোর্ট করে না। তার পরিবর্তে, কমেন্ট এর জন্য সিমুলেটেড কী বা ভ্যালু বা একটি স্ট্রিং ব্যবহার করা হয়।

{
  "_comment": "This is a comment",
  "name": "Bob"
}

6. কনভারসন:
JSON এবং PHP ডেটা টাইপ মডিউল মধ্যে কনভার্সন হতে পারে json_encode() এবং json_decode() ফাংশন ব্যবহার করে।

                    ========================= PHP ===========================

জেনারেট করা (JSON Encode):
PHP এবং অন্যান্য ভাষাতে ডেটা টাইপকে JSON স্ট্রিং এ কনভার্ট করতে json_encode() ফাংশনটি ব্যবহার করা হয়।

  <?php 

    $data = array(
        "name" => "John",
        "age" => 30,
        "city" => "New York"
    );

    $json_string = json_encode($data);
    echo $json_string;
  ?>

উপরের কোডে, একটি অ্যাসোসিয়েটিভ অ্যারেকে JSON স্ট্রিং এ রুপান্তরিত করতে json_encode() ব্যবহৃত হয়েছে।


পার্স করা (JSON Decode):
JSON স্ট্রিং থেকে ডেটা টাইপকে পার্স করতে json_decode() ফাংশনটি ব্যবহার করা হয়।

<?php 

    $json_string = '{"name":"John", "age":30, "city":"New York"}';

    $data = json_decode($json_string);
    print_r($data);


?>

উপরের কোডে, JSON স্ট্রিং থেকে ডেটা পার্স করতে json_decode() ফাংশনটি ব্যবহৃত হয়েছে।
JSON এর এই বিশেষ ফরম্যাট দ্বারা ডেটা টাইপের মধ্যে সহজেই রুপান্তরিত করা যায় এবং তা হুম্যান রিডেবল এবং সহজ হয়ে থাকে যা ওভার দিওয়া এবং পাঠানো সহজ।