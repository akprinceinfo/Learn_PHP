
JSON (JavaScript Object Notation) হলো একটি লাইটওয়েট ডেটা ইনটারচেঞ্জ ফরম্যাট, যা মানুষ ও মেশিনদের জন্য সহজে বুঝা এবং তৈরি করা যায়। এটি টেক্সট ফরম্যাট এবং কোড এবং ডেটা এর মধ্যে একটি একক স্ট্যান্ডার্ড ইন্টারফেস প্রদান করে। JSON স্ট্রিং হিসেবে রুপান্তরিত করা যায় যেখানে ডেটা গুলি অবজেক্ট বা অ্যারে হিসেবে থাকতে পারে।

বিস্তারিত বর্ণনা:
1. ডেটা টাইপ:
JSON তে প্রিমিটিভ ডেটা টাইপ গুলি হলো -

স্ট্রিং (String): টেক্সট। উদাহরণ: "Hello, World!"
সংখ্যা (Number): কোনো সংখ্যা। উদাহরণ: 42 অথবা 3.14
বুলিয়ান (Boolean): true অথবা false
নাল (Null): null

2. অবজেক্ট (Object):
অবজেক্ট হলো কী-ভ্যালু এর জোড়াসহ একটি কলেকশন। কী হলো একটি স্ট্রিং এবং এটি ভ্যালুকে সাথে জোড়া করে। অবজেক্টের শুরু হয় '{' এবং শেষ হয় '}' এর মধ্যে। কলন সাথে একটি অবস্থানের সিমুলেট হয়, সেটি নিজেই একটি অবজেক্ট হিসেবে ফাঁকা থাকতে পারে। উদাহরণ:

{
  "name": "John",
  "age": 30,
  "city": "New York"
}

3. অ্যারে (Array):
অ্যারে হলো একটি ডেটা কলেকশন যা সিরিজে এবং ইনডেক্সড ডেটা হিসেবে থাকে। অ্যারের শুরু হয় '[' এবং শেষ হয় ']' এর মধ্যে। উদাহরণ:

["apple", "orange", "banana"]

