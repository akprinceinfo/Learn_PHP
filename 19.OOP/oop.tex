php Constructor & Destructor কেনো ব্যবহার করা হয় ?

PHP ক্লাসে Constructor এবং Destructor দুটি বিশেষ মেথড যা ক্লাসের অবজেক্ট তৈরি হওয়ার সময় এবং অবজেক্ট ধ্বংস হওয়ার সময় স্বয়ংক্রিয়ভাবে কাজ করে।

Constructor (কনস্ট্রাক্টর):
কনস্ট্রাক্টর ক্লাসের অবজেক্ট তৈরির সময় স্বয়ংক্রিয়ভাবে কাজ করে।
কনস্ট্রাক্টরের নাম সাধারণভাবে __construct হয়।
কনস্ট্রাক্টর একটি ক্লাসে একবার মাত্র থাকতে পারে।
কনস্ট্রাক্টর অবজেক্ট তৈরির সাথে সাথে অটোমেটিকভাবে কল হয়।

% <?php
%     class MyClass {
%         public function __construct() {
%             echo "Constructor called!";
%         }
%     }

%     $obj = new MyClass(); // Output: Constructor called!

% ?>


Destructor (ডিসট্রাক্টর):
ডিসট্রাক্টর অবজেক্ট ধ্বংস হওয়ার সময় স্বয়ংক্রিয়ভাবে কাজ করে।
ডিসট্রাক্টরের নাম সাধারণভাবে __destruct হয়।
এটি একটি ক্লাসে একবার মাত্র থাকতে পারে।
ডিসট্রাক্টর অটোমেটিকভাবে কল হয়, যখন অবজেক্ট ধ্বংস হয়।


    % class MyClass {
    %     public function __destruct() {
    %         echo "Destructor called!";
    %     }
    % }

    % $obj = new MyClass();
    % unset($obj); // Output: Destructor called!


    কনস্ট্রাক্টর এবং ডিসট্রাক্টর ব্যবহার করা হয় যেন অবজেক্ট তৈরির সময় নির্দিষ্ট অবস্থা সেটআপ করা যায় এবং অবজেক্ট ধ্বংস হওয়ার সময় নির্দিষ্ট রিসোর্স রিলিজ করা যায়।






