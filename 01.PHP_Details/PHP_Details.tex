PHP : Php is Surver side Scripting Language.

পিএইচপি (PHP) হল একটি সার্ভার-সাইড স্ক্রিপ্টিং ভাষা, যা মূলত ওয়েব ডেভেলপমেন্টে ব্যবহার করা হয়। পিএইচপি কোড সার্ভারে চালিত হয় এবং ব্রাউজারে প্রদর্শনের জন্য HTML হিসেবে প্রেরণ করা হয়।

পিএইচপি কিভাবে কাজ করে, সে সম্পর্কে সংক্ষেপে জানতে হলে:

1.Client request / ক্লায়েন্ট অনুরোধ: যখন আপনি কোন ওয়েবসাইটে ব্রাউজ করেন, আপনি সেই সাইটের সার্ভারে একটি অনুরোধ প্রেরণ করেন।

2.Server processing/সার্ভার প্রক্রিয়াজাতকরণ: সার্ভারে থাকা PHP প্রোগ্রাম চালানো হয়। পিএইচপি কোডটি এখানে প্রক্রিয়াজাত করা হয় এবং HTML কোড তৈরি হয়।

3.Sending Results/ফলাফল প্রেরণ: এই HTML কোডটি ক্লায়েন্টে (অর্থাৎ আপনার ব্রাউজারে) প্রেরণ করা হয়। আপনার ব্রাউজার এই HTML কোড দেখে এবং ওয়েবপেজটি প্রদর্শন করে।

4.display/প্রদর্শন: আপনি প্রাথমিকভাবে পিএইচপি কোড দেখেন না, শুধু তার উৎপাদিত HTML আউটপুট দেখেন।

পিএইচপি কাজ করার জন্য সার্ভারে পিএইচপি ইন্সটল করা থাকতে হবে। এছাড়া, এটি ডাটাবেস যেমন MySQL সাথে যোগাযোগ করতে পারে, তাদের মধ্যে ডাটা পাঠানো এবং গ্রহণ করতে পারে। প্রয়োজনে, একটি লোকাল ডেভেলপমেন্ট পরিবেশ যেমন XAMPP বা WAMP ব্যবহার করে আপনি নিজের কম্পিউটারে পিএইচপি কোড লেখা ও পরীক্ষা করতে পারেন।
